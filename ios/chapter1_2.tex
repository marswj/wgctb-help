\subsection{错题本}
现在的主界面如图(\ref{img18})
\begin{figure}[H]
	\centering
	\includegraphics{img/18.png}
	\caption{主界面}
	\label{img18}
\end{figure}
现在在小红的错题本“一年级数学(下)”里面添加错题,点击“一年级数学(下)”条目进入错题本,如图(\ref{img19})
\begin{figure}[H]
	\centering
	\includegraphics{img/19.png}
	\caption{一年级数学(下)}
	\label{img19}
\end{figure}
此页面显示了错题本名、学生名、错题数量、复习进度(已经复习/错题数量)、存储空间。

右上角是几个操作按钮、分别是相册选取、拍照、复习、重置。

\subsubsection{增加错题}
步骤1:可以选择“相册选取”或“拍照”获得一张错题照片。建议手机横拍。这里需要权限(照片、相机),请允许。
\begin{itemize}
	\item 相册选取(建议,效率更高。先集中一起拍照,然后从相册中一张张选择处理)
	\item 拍照(拍一张,处理一张)
\end{itemize}

步骤2:把照片裁剪到合适的大小,如图(\ref{img20})。满意点击右上角OK 。不满意点击左上角Back退出裁剪。

\begin{figure}[H]
	\centering
	\includegraphics{img/20.png}
	\caption{裁剪}
	\label{img20}
\end{figure}

步骤3:涂改掉多余的答案等,仅保留题干,如图(\ref{img21})。满意点击右上角OK 。不满意点击左上角Back退出涂改。
\begin{figure}[H]
	\centering
	\includegraphics{img/21.png}
	\caption{涂改}
	\label{img21}
\end{figure}

步骤4:点击右上角Done,进入如图(\ref{img22}),就是复习时会看到的错题。可选操作,右上角有3个按钮,分别是错解、正解、总结,可以为这道错题添加上错误的解法、正确的解法、自己的归纳总结等(还是通过“拍照→剪切→涂改→保存”这个步骤录入。)。如果不需要额外增加信息,可以直接点击左上角回退到错题本。
\begin{figure}[H]
	\centering
	\includegraphics{img/22.png}
	\caption{错题浏览}
	\label{img22}
\end{figure}

现在的错题本看上去是这样,如图(\ref{img23})
\begin{figure}[H]
	\centering
	\includegraphics{img/23.png}
	\caption{错题本}
	\label{img23}
\end{figure}

\subsection{复习}
\begin{figure}[H]
	\centering
	\includegraphics{img/24.png}
	\caption{进入复习}
	\label{img24}
\end{figure}

点击复习按钮,如图(\ref{img24}),进入复习页面,如图(\ref{img25})

\begin{figure}[H]
	\centering
	\includegraphics{img/25.png}
	\caption{复习}
	\label{img25}
\end{figure}

随机从未复习的错题中选择一题,根据本次答题情况,分别选择右上角菜单中的”正确“或”错误“ 按钮。回答正确的,本轮复习中不会再次出现。

右上角下拉菜单中,有错解、正解、总结按钮,可以这时增加当前错题的错解、正解、总结。移动按钮可以移动当前错题到其他错题本中。删除按钮,可以删除当前错题。

\subsection{重置}
\begin{figure}[H]
	\centering
	\includegraphics{img/26.png}
	\caption{重置}
	\label{img26}
\end{figure}
错题本中的错题全部复习完成,点击如图(\ref{img26})按钮可以重新重置错题状态为可复习状态。